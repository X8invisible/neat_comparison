%You can delete all the comments after you have finished your document
%this sets up the defaults for the documents, 12pt font and A4 size. The article type sets this up as such as opposed to letter or memo.

%for the finer points LaTeX see https://en.wikibooks.org/wiki/LaTeX or http://tex.stackexchange.com/

\documentclass[12pt,a4paper]{article}
\usepackage{titlesec} %these are how we import packages, one helps set up footers and title layout
\usepackage{fancyhdr}

% !TEX TS-program = pdflatex
% !TEX encoding = UTF-8 Unicode
\usepackage[utf8]{inputenc} % set input encoding (not needed with XeLaTeX)

\usepackage{graphicx} % support the \includegraphics command and options

% \usepackage[parfill]{parskip} % Activate to begin paragraphs with an empty line rather than an indent

%%% PACKAGES
\usepackage{booktabs} % for much better looking tables
\usepackage{array} % for better arrays (eg matrices) in maths
\usepackage{paralist} % very flexible & customisable lists (eg. enumerate/itemize, etc.)
\usepackage{verbatim} % adds environment for commenting out blocks of text & for better verbatim
\usepackage{subfig} % make it possible to include more than one captioned figure/table in a single float
\usepackage[toc,page]{appendix}
% These packages are all incorporated in the memoir class to one degree or another...

%header and footer settings
\pagestyle{fancyplain}
\fancyhf{}
\renewcommand{\headrulewidth}{0.5pt}
\renewcommand{\footrulewidth}{0.5pt}
\setlength{\headheight}{15pt}
\fancyhead[L]{Ovidiu-Andrei Radulescu - 40283288}
\fancyhead[R]{ SOC10101 Honours Project}
\fancyfoot[L]{}
\fancyfoot[C]{\thepage}

%set better section layout
\makeatletter
\renewcommand\subsection{\@startsection {subsection}{1}{2mm} % name, level, indent
                               {3pt plus 2pt minus 1pt} % before skip
                               {3pt plus 0pt} % after skip
                               {\normalfont\bfseries}}
\makeatother
\makeatletter
\renewcommand\section{\@startsection {section}{1}{0mm} % name, level, indent
                               {4pt plus 2pt minus 1pt} % before skip
                               {4pt plus 0pt} % after skip
                               {\bfseries}}
\makeatother


%this starts the document
\begin{document}

%you can import other documents into your main one, these layout the Title and Declarations on its own page.
%you might need to change these to \ if your on Microsoft Windows.
\input{./Dissertation-Title.tex}
\input{./Dissertation-Dec.tex}
\pagebreak
\input{./Dissertation-DP.tex}
\pagebreak

%LaTeX let you define the abstract separately so it wont get sucked into the main document.
\begin{abstract}
% fill the abstract in here
test
\end{abstract}
\pagebreak

\tableofcontents % is generated for you
\newpage

\listoftables
%generated in same way as figures
\newpage

\listoffigures
%you may have captions such as equations, listings etc they should all appear as required
%these are done for you as long as you use \begin{figure}[placement settings] .. bla bla ... \end{figure}
\newpage


\section{Introduction}
This piece of writing aims to investigate different Artificial Intelligence (AI) methods used in the training of Neural Networks (NN or NNs).\par
Artificial Intelligence (AI) is intelligence demonstrated by machines, as opposed to natural intelligence shown by humans and other animals. As computers started progressing and becoming better and better at numerical calculations, their use-cases became more complex with the employment of algorithms, which are sequences of actions that resemble the way a human mind would approach a problem, thus making a computer exhibit certain capabilities of the human mind (Wang, 2007).\par
A Neural Network (NN), sometimes called an Artifical NN, is an interconnected system formed from simple processing elements, inspired by (but not identical to) a biological brain. Such a system learns how to perform a task by adapting or learning from a set of training patterns. This processing ability of the network is "stored in the interunit connection strengths,or weights." (Gurney, 2014).\par
To understand how best to attempt to create multiple Neural Networks that will ultimately be able to solve tasks, this essay aims to: define the meaning of a Neural Network, as well as the Artificial Inteligence methods used to train a NN; breakdown the different kinds of training methods to find the advantages (and disadvantages) of each, depending on their use case; look into the various applications of Neural Networks and provide the technological context in the modern day by looking at the history and the development, both past, present and future; conduct an investigation into existing environments used as benchmarks for a Neural Network's accuracy and efficiency; make a decision as to which environments are best suited for use within this project; consider how this project could be expanded upon in future developments.



\newpage
\begin{thebibliography}{9}

\bibitem{lamport94}
  Leslie Lamport,
  \emph{\LaTeX: A Document Preparation System}.
  Addison Wesley, Massachusetts,
  2nd Edition,
  1994.

\end{thebibliography}
%example of References. See https://en.wikibooks.org/wiki/LaTeX/Bibliography_Management
%might be good to use a separate document for these so your main work is not one really long text file. 

%you can crate this on a extra tex document just like the title or any other part of the document.
\newpage
\begin{appendices}
\section{Project Overview}
%insert IPO

\begin{subappendices}
\subsection{Example sub appendices}
...
\end{subappendices}

\section{Second Formal Review Output}
Insert a copy of the project review form you were given at the end of the review by the second marker

\section{Diary Sheets (or other project management evidence)}
Insert diary sheets here together with any project management plan you have

\section{Appendix 4 and following}
insert content here and for each of the other appendices, the title may be just on a page by itself, the pages of the appendices are not numbered, unless an included document such as a user manual or design document is itself pager numbered.
\end{appendices}

\end{document}
